\documentclass[a4paper,11pt,dvipsnames]{article}
\usepackage{hyperref}
\usepackage{amsmath}
\usepackage{titling}  %allows for title and abstract to be together. There are also packages: title and abstract.
\usepackage[margin = 1in]{geometry}
\usepackage{xcolor}

\title{\textbf{Phase Field Solver for Recrystallization}}
\author{Vitesh Shah}

\begin{document}
\nonstopmode
\begin{titlingpage}
	\maketitle
	\begin{abstract}
	This document details the solving strategy for phase field solvers being used for recrystallization. 
    Here there is no details of the proof of the equations. 
    Refer to document `Phase\textunderscore field\textunderscore methods' for details of the proofs. 
    The methods described here are borrowed from work of Shantharaj et al. 
	\end{abstract}
\end{titlingpage}

%======================================================================================================================================================================================
%-------------Describing new commands for some special mathematical characters---------------------------------------------------------------------------------------------------------
\newcommand\mathsym[1]{\begin{math} {#1} \end{math}}
\newcommand{\Tau}{\mathrm{T}}
%====================================================================================================================================================================================

\section{Orientation field solver}
The equation for the orientation field can be described as equation 40 in \cite{Abrivard2012b},

\begin{equation}
    \tau_\theta \frac{\partial \theta}{\partial t} = \nabla \cdot \left ( 2h(\eta) \nabla \theta + g(\eta) \frac{\nabla \theta}{\left | \nabla \theta \right |} \right ) \label{eq:1} 
\end{equation}

For initial attempt, we simplify the equation above,
\begin{equation}
    \tau_\theta \frac{\partial \theta}{\partial t} = \nabla \cdot \left ( 2h(\eta) \nabla \theta \right ) \label{eq:2} 
\end{equation}

If we assume that \mathsym{\theta} is a scalar and 
using backward euler and splitting into fluctuations, the above equation can be written as, 

\begin{equation}
\left ( \overline{\tau}_\theta + \tilde{\tau}_\theta \right ) \frac{\theta (x,t) - \theta (x,t_o)}{\Delta t} = \nabla \cdot \left [ 2 \overline{h}(\eta) \nabla \theta (x,t) - \tilde{T} (x) \right ] \label{eq:3}
\end{equation}

The equation above can be re-written in the Fourier space as,

\begin{equation}
\left ( \overline{\tau}_\theta + \tilde{\tau}_\theta \right ) \frac{\theta (k,t) - \theta (k,t_o)}{\Delta t} = \mathcal{F} \nabla \cdot \left [ 2 \overline{h}(\eta) \nabla \theta (x,t) - \tilde{T} (x) \right ] \label{eq:4}
\end{equation}

Expressing \mathsym{\theta (x,t)} in Fourier space,  

\begin{equation}
\mathcal{F} \left [ \nabla \cdot \left [ 2 \overline{h}(\eta) \nabla \theta (x,t) \right ] \right ] = 
\mathcal{F} \left [ \nabla \cdot \left [ 2 \overline{h}(\eta) \right ] \left ( \nabla \mathcal{F}^{-1} \left (\theta (k,t) \right ) \right ) \right ] \label{eq:5}
\end{equation}

where \mathsym{\theta (x,t) = \mathcal{F}^{-1} \left (\theta (k,t) \right )}. 
Using definition of inverse fourier transform, 

\begin{equation}
\mathcal{F} \left [ \nabla \cdot \left [ 2 \overline{h}(\eta) \nabla \theta (x,t) \right ] \right ] = 
\mathcal{F} \left [ \nabla \cdot \left [ 2 \overline{h}(\eta) \right ] \left ( \nabla \int \theta (k,t) e^{ikx} dk \right ) \right ] \label{eq:6}
\end{equation}

The gradient operator leads to following integral, 

\begin{equation}
\mathcal{F} \left [ \nabla \cdot \left [ 2 \overline{h}(\eta) \nabla \theta (x,t) \right ] \right ] = 
\mathcal{F} \left [ \nabla \cdot \left [ 2 \overline{h}(\eta) \right ] \left ( \int ik \theta (k,t) e^{ikx} dk \right ) \right ] \label{eq:7}
\end{equation}

\begin{equation}
\mathcal{F} \left [ \nabla \cdot \left [ 2 \overline{h}(\eta) \nabla \theta (x,t) \right ] \right ] = 
\mathcal{F} \left [ \nabla \cdot \left ( \int 2 \overline{h} (\eta) \cdot ik \cdot \theta (k,t) e^{ikx} dk \right ) \right ] \label{eq:8}
\end{equation}

\begin{equation}
\mathcal{F} \left [ \nabla \cdot \left [ 2 \overline{h}(\eta) \nabla \theta (x,t) \right ] \right ] = 
\mathcal{F} \left [ \int 2 \overline{h} (\eta) \cdot ik \cdot \theta (k,t) \cdot ik \cdot e^{ikx} dk \right ] \label{eq:9}
\end{equation}

Again invoking the definition of inverse fourier transform,

\begin{equation}
\mathcal{F} \left [ \nabla \cdot \left [ 2 \overline{h}(\eta) \nabla \theta (x,t) \right ] \right ] = 
\mathcal{F} \left [ \mathcal{F}^{-1} \left ( 2 \overline{h} (\eta) \cdot ik \cdot \theta (k,t) \cdot ik \right ) \right ] \label{eq:10}
\end{equation}

\begin{equation}
\mathcal{F} \left [ \nabla \cdot \left [ 2 \overline{h}(\eta) \nabla \theta (x,t) \right ] \right ] = 
\left ( 2 \overline{h} (\eta) \cdot ik \cdot \theta (k,t) \cdot ik \right ) \label{eq:11}
\end{equation}

Similarly, the second term in right hand side can be expressed as, 

\begin{equation}
\mathcal{F} \left [ \nabla \cdot \tilde{T}(x) \right ] = 
\tilde{T}(k) \cdot ik \label{eq:12}
\end{equation}

Therefore, the equation \ref{eq:4} can be written in Fourier space as, 

\begin{equation}
\left ( \overline{\tau}_\theta + \tilde{\tau}_\theta \right ) \frac{\theta (k,t) - \theta (k,t_o)}{\Delta t} = \left ( 2 \overline{h} (\eta) \cdot ik \cdot \theta (k,t) \cdot ik \right ) +  \tilde{T}(k) \cdot ik \label{eq:13}
\end{equation}

\begin{equation}
\left ( \overline{\tau}_\theta + \tilde{\tau}_\theta \right )\left ( \theta (k,t) - \theta (k,t_o)\right ) = \Delta t \left ( 2 \overline{h} (\eta) \cdot ik \cdot \theta (k,t) \cdot ik +  \tilde{T}(k) \cdot ik \right ) \label{eq:14}
\end{equation}

Re-arranging the terms leads to, 
\begin{equation}
\left ( \overline{\tau}_\theta + \Delta t \cdot k \cdot 2 \overline{h} (\eta) \cdot k \right ) \theta (k,t) = 
\Delta t \tilde{T}(k) \cdot ik + \overline{\tau}_\theta (k,t_o) - \tilde{\tau}_\theta (k,t) + \tilde{\tau}_\theta (k,t_o) \label{eq:14}
\end{equation}

where, \mathsym{\tilde{T}(k) \approx 2 h (\eta) - D_{\text{ref}}}. 
\mathsym{D_{\text{ref}}} is assumed to be the constant \mathsym{2 h (\eta)} calculated at the very initial time step. 
The term on the left hand side can be defined as the gamma convolution, 

\begin{equation}
    \Gamma^{-1} = \overline{\tau}_\theta + \Delta t \cdot k \cdot 2 \overline{h} (\eta) \cdot k \label{eq:15}
\end{equation}

This leads to following residual, 

\begin{equation}
    \theta (k,t) - \Gamma (k) T(k) = 0 \label{eq:16}
\end{equation}

where T(k) is the entire expression on the right hand side of equation \ref{eq:14}. 

\bibliographystyle{ieeetran}
%can use bibliographystyle{plain} for the most basic things but it doesnt order the citations properly
\bibliography{library}
\end{document}

